% \documentclass{beamer}
% \usepackage[utf8]{inputenc}
% \usepackage{graphicx}
% \usepackage{amsmath}
% \usepackage{booktabs}
% !TEX TS-program = xelatex

% HSE Beamer Theme
% Version 2.0 (English)
% January 2022

\documentclass[aspectratio=169]{beamer}
\newbool{russian}
%\booltrue{russian} % Uncomment if in Russian
\usepackage{HSE-theme/beamerthemeHSE} % Load HSE theme

\usepackage[no-math]{fontspec}      % fonts loading
        \setsansfont{HSE Sans}
        \setmonofont{Courier New}
\usepackage{mathspec}
        \setmathsfont(Digits,Latin,Greek)[Numbers={Lining,Proportional}]{HSE Sans}
        \setmathrm[Numbers={Lining,Proportional}]{HSE Sans}

\graphicspath{{images/}}        % Images folder



\title{Прогнозирование вязкости жидкостей с использованием машинного обучения на основе параметров кубических уравнений состояния}
% \subtitle{}
\author{Панов Михаил}
\date{\today}

\begin{document}

\begin{frame}
    \titlepage
\end{frame}

\begin{frame}{Проблема, объект и предмет исследования}
    \textbf{Проблема:} Точность предсказания вязкости жидкостей, состоящих из простых углеводородов.\\
    \textbf{Объект:} Вязкость жидких смесей углеводородов в различных термодинамических условиях.\\
    \textbf{Предмет:} Методы прогнозирования вязкости жидкостей, включая аналитические модели и подходы машинного обучения.
\end{frame}

\begin{frame}{Результаты}
    \begin{itemize}
        \item Проанализирована база данных ThermoML на предмет соответствующих данных.
        \item Собраны экспериментальные данные по вязкости.
        \item Данные дополнены коэффициентами веществ в уравнении CPPCSAFT.
        \item Вычислены плотности для экспериментальных данных на основе давления и температуры.
        \item Посчитана избыточная энтропия с помощью полученной плотности.
    \end{itemize}
\end{frame}

\begin{frame}{Результаты}
    \begin{itemize}
        \item Найдены члены, характерные для формул вязкости.
        \item Данные дополнены производными признаками, дающими наибольший вклад в формулы.
        \item Обучены символьные модели.
        \item Обучены модели машинного обучения, предсказывающие вязкость жидкости с помощью экспериментальных данных, коэффициентов уравнения состояния и вычисленных параметров.
        \item Произведено сравнение точности полученных моделей с методами масштабирования энтропии и расширенной жидкостной корреляции.
    \end{itemize}
\end{frame}

\begin{frame}{Задачи}
    \begin{itemize}
        \item Сбор и обработка экспериментальных данных по вязкости из базы данных ThermoML.
        \item Вычисление дополнительных параметров с использованием термодинамических соотношений и уравнений состояния.
        \item Определение оптимальных параметров, включая их преобразования и комбинации, для повышения точности прогнозирования вязкости.
        \item Разработка и проверка моделей машинного обучения, которые прогнозируют вязкость на основе имеющихся параметров.
        \item Сравнение моделей ML с известными методами.
    \end{itemize}
\end{frame}

\begin{frame}{Цель исследования}
    Разработать метод для прогнозирования вязкости жидкости на основе параметров кубических уравнений состояния (CubicEOS), повышающий точность прогнозирования по сравнению с классическими методами.
\end{frame}

\end{document}


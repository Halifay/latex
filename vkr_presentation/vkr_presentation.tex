%!TEX TS-program = xelatex

% HSE Beamer Theme
% Version 2.0 (English)
% January 2022

\documentclass[aspectratio=169]{beamer}
\newbool{russian}
%\booltrue{russian} % Uncomment if in Russian
\usepackage{HSE-theme/beamerthemeHSE} % Load HSE theme

\usepackage[no-math]{fontspec}      % fonts loading
	\setsansfont{HSE Sans}
	\setmonofont{Courier New}
\usepackage{mathspec}
	\setmathsfont(Digits,Latin,Greek)[Numbers={Lining,Proportional}]{HSE Sans}
	\setmathrm[Numbers={Lining,Proportional}]{HSE Sans}

\graphicspath{{images/}}  	% Images folder

% \titlegraphic{\includegraphics[width=2cm]{logo.png}} % по желанию

\begin{document}

\begin{frame}
    \centering
    \vspace{1cm}
    {\Large\textbf{Прогнозирование вязкости жидкостей с использованием машинного обучения\\ на основе параметров кубических уравнений состояния}}\\[2cm]

    {\bfseries Студент-исследователь:}\\
    Панов Михаил Федорович\\
    студент 2 курса магистратуры ОП\\
    «Системный анализ и математические технологии»\\[1.5cm]

    {\bfseries Руководитель проекта:}\\
    Писарев Василий Вячеславович\\
    ведущий научный сотрудник\\
    Международная лаборатория САММА\\[1.5cm]

    \vfill
    \begin{flushright}
    Москва, 2025
    \end{flushright}
\end{frame}

\begin{frame}
    \titlepage
    \vfill
    \begin{flushright}
    Руководитель: ФИО, степень, звание, должность \\
    Консультант: ФИО, если есть
    \end{flushright}
\end{frame}

\begin{frame}{Постановка задачи}
    \textbf{Цель:} Разработка метода для прогнозирования вязкости жидкости на основе параметров CubicEOS, повышающего точность по сравнению с классическими методами.\\[1ex]
    \textbf{Задачи:}
    \begin{enumerate}
        \item Сбор и обработка экспериментальных данных из базы ThermoML.
        \item Вычисление дополнительных параметров с использованием термодинамических соотношений и уравнений состояния.
        \item Определение оптимальных параметров и их комбинаций.
        \item Разработка и проверка моделей машинного обучения.
        \item Сравнение моделей с классическими методами.
    \end{enumerate}
\end{frame}

\begin{frame}{Анализ и обоснование выбора методов}
    \begin{itemize}
        \item Классические методы (энтропийное масштабирование, жидкостные корреляции) имеют ограниченную точность.
        \item Использование CPPCSAFT позволяет рассчитывать термодинамические параметры.
        \item Машинное обучение обеспечивает более высокую гибкость и точность.
        \item Генерация новых признаков на основе формул помогает улучшить качество модели.
    \end{itemize}
\end{frame}

\begin{frame}{Сбор и обработка данных}
    \begin{itemize}
        \item ThermoML использована как источник экспериментальных данных.
        \item Данные приведены к единому формату, очищены и дополнены коэффициентами CPPCSAFT.
        \item Расчёт плотностей по температуре и давлению.
    \end{itemize}
\end{frame}

\begin{frame}{Расчёт параметров и энтропии}
    \begin{itemize}
        \item Вычисление избыточной энтропии на основе плотности.
        \item Анализ производных термодинамических величин.
        \item Поиск характеристических членов формул для вязкости.
    \end{itemize}
\end{frame}

\begin{frame}{Генерация и отбор признаков}
    \begin{itemize}
        \item Реализована собственная система генерации признаков.
        \item Используются линейная регрессия и лес деревьев для отбора важных признаков.
        \item Генерация новых формул и отбор наиболее значимых.
    \end{itemize}
\end{frame}

\begin{frame}{Разработка моделей}
    \begin{itemize}
        \item Построены и обучены модели машинного обучения (регрессия, деревья).
        \item Используются как экспериментальные данные, так и вычисленные признаки.
        \item Многократная валидация и тестирование моделей.
    \end{itemize}
\end{frame}

\begin{frame}{Сравнение с классическими методами}
    \begin{itemize}
        \item Масштабирование энтропии.
        \item Расширенные жидкостные корреляции.
        \item Метод с генерацией признаков даёт улучшение точности в несколько раз.
    \end{itemize}
\end{frame}

\begin{frame}{Результаты}
    \begin{itemize}
        \item Получены высокоточные модели для прогнозирования вязкости.
        \item Создана система отбора признаков с возможностью расширения.
        \item Результаты превосходят традиционные подходы.
    \end{itemize}
\end{frame}

\begin{frame}{Выводы и перспективы}
    \textbf{Выводы:}
    \begin{enumerate}
        \item Собраны и обработаны данные из ThermoML.
        \item Вычислены важные параметры на основе уравнений состояния.
        \item Проведена генерация и отбор признаков.
        \item Построены и проверены модели машинного обучения.
        \item Модели превзошли классические методы.
    \end{enumerate}
    \textbf{Перспективы:}
    \begin{itemize}
        \item Расширение на другие классы жидкостей.
        \item Интеграция в инженерные расчётные системы.
        \item Развитие автоматизированной системы генерации признаков.
    \end{itemize}
\end{frame}

\end{document}


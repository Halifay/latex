\documentclass[a4paper,12pt]{article}

\usepackage{fontspec}
\usepackage{polyglossia}
\setmainlanguage{russian}
\setotherlanguage{english}

\setmainfont{Times New Roman}
\newfontfamily\cyrillicfont{Times New Roman}
\newfontfamily\cyrillicfonttt{Courier New}

\usepackage{amsmath, amssymb, amsfonts, csquotes, array, geometry, titlesec, listings, hyperref, caption, xcolor, import, subcaption}

% Numbering of only mentioned formulas
\usepackage{mathtools}
\mathtoolsset{showonlyrefs}

\usepackage{enumitem}
\usepackage{alphabeta} % Optional: for Greek if needed

\makeatletter
\AddEnumerateCounter{\asbuk}{\russian@asbuk@alph}{щ} % section 8.1
\makeatother

% Define Cyrillic enumeration using custom labels
\setlist[enumerate, 1]{label=\asbuk*)}

% Changing itemize dots to dashes
\setlist[itemize]{label=-}

\usepackage{graphicx}
\graphicspath{{images/}}  % Presets the "images" directory

\usepackage[backend=biber, style=numeric-comp, sorting=none]{biblatex}
\addbibresource{references.bib} % Файл со списком литературы

% Для управления таблицами и правильной нумерации
\usepackage{caption}            % Настройка подписей
\captionsetup[table]{justification=raggedright, labelsep=period, textfont=normalsize}
% \captionsetup[table]{font=normalsize, labelfont=bf, labelsep=period, justification=raggedright}

% Для изменения шрифта в таблицах
\usepackage{array}

% Для многострочных заголовков и гибкой ширины колонок
\usepackage{makecell}
\renewcommand\theadfont{\bfseries}

% Patch caption to store the number only
\makeatletter
\newcommand{\TableNumberRight}{
  \refstepcounter{table}% increment and make labelable
  \noindent\hfill\textbf{Таблица \thetable}
}
\makeatother

\usepackage{etoolbox}
\AtBeginEnvironment{tabular}{\small}  % or \footnotesize, \scriptsize

\usepackage{siunitx}
\sisetup{
  group-separator     = {\,},   % optional: use space between thousands
  detect-all,                 % makes font/style match surroundings
  table-align-text-post = false
}

\geometry{left=25mm,right=20mm,top=25mm,bottom=25mm}

\hypersetup{
	% linkcolor=white,
	linkbordercolor = white,          
	citebordercolor = green,
}

\setcounter{page}{2}

\title{\Huge{Предсказание вязкости жидкостей с помощью методов машинного обучения на основе параметров уравнения состояния}}
\author{Панов Михаил Федорович \\ Руководитель: Писарев Василий Вячеславович}
\date{Москва --- 2025}

\begin{document}

% Титульный лист
\maketitle
\thispagestyle{empty}
\newpage

% Задание на ВКР
\section*{Задание на ВКР}
(Текст задания на ВКР)

% Аннотация
\section*{Аннотация}
(Аннотация на русском языке)

% Abstract
\section*{Abstract}
(Аннотация на английском языке)
\newpage

% Содержание
\tableofcontents
\newpage

% Введение
\section*{Введение}
\addcontentsline{toc}{section}{Введение}
(Введение: актуальность, цель, задачи, объект, структура)

% Глава 1
\section{Обзор литературы и постановка задачи}
  \subsection{Свойства вязкости и ее роль в инженерных задачах}

    \subsubsection{Вязкость в промышленных приложениях}

Вязкость — ключевой реологический параметр, определяющий сопротивление жидкости деформации при сдвиге. Ее значение напрямую влияет на гидродинамические характеристики систем, включая распределение давления, скорости потока и теплопередачу. В ряде инженерных процессов, таких как транспортировка нефти и газа, проектирование теплообменников, химическая переработка и производство смазочных материалов, точное знание вязкости определяет эффективность и безопасность эксплуатации оборудования. 

В многофазных или сложных многокомпонентных системах, например, в синтетических нефтяных смесях, вязкость влияет на фазовое поведение, устойчивость эмульсий и общую производительность процессов. В микрофлюидике и фармацевтике контроль вязкости необходим для предсказуемого управления потоками и дозированием веществ.

    \subsubsection{Значение точного прогнозирования вязкости}

Точное моделирование вязкости представляет собой одну из ключевых задач в термодинамическом моделировании жидкостей. На практике невозможно измерить вязкость для всех возможных условий и составов, особенно в системах, содержащих десятки или сотни компонентов, как, например, в нефтяной промышленности. В таких случаях используются предсказательные модели, основанные на молекулярных или термодинамических представлениях, которые позволяют получить значения вязкости при отсутствии экспериментальных данных.

Среди современных подходов особое внимание уделяется моделям, связывающим вязкость с остаточной энтропией и другими термодинамическими величинами, так как они обеспечивают высокую переносимость и физическую обоснованность. Связь между вязкостью и избыточной энтропией подмечали во многих работах, например \cite{taib2020residual}. Особенно актуальны такие модели для расчетов в широком диапазоне температур и давлений, а также при переходе к сверхкритическим и конденсированным фазам.

Таким образом, разработка универсальных моделей вязкости, не требующих индивидуальной подгонки параметров, является приоритетной задачей в области физико-химического моделирования сложных жидкостей.

  \subsection{Обзор подходов к предсказанию вязкости}

    \subsubsection{Ограничения эмпирических формул}
      Большинство классических подходов к описанию вязкости основаны на эмпирических уравнениях, содержащих подобранные коэффициенты для конкретных веществ или их классов. Такие модели, как правило, работают в узком диапазоне температур и давлений и требуют предварительных измерений вязкости. Это делает их трудноприменимыми для новых или редких веществ, особенно в условиях, когда экспериментальные данные недоступны или ограничены. (for example)

    \subsubsection{Модель MYS}

      Одной из наиболее универсальных и известных современных моделей для описания динамической вязкости жидкостей в широком диапазоне условий является модель, предложенная Ильей Полищуком в 2015 году — так называемая \textit{Modified Yarranton–Satyro (MYS)} модель~\cite{polishuk2015viscosity}. Ее основное назначение — обеспечить точное предсказание вязкости для широкого спектра органических жидкостей, включая углеводороды, в диапазонах температур от тройной точки до высоких значений, а также при изменяющемся давлении.

      \textbf{Предпосылки модели.}  
      Изначально модель Yarranton–Satyro (YS) создавалась для описания вязкости как функции плотности, температуры и параметров молекулярного строения. Polishuk модифицировал ее, чтобы избежать необходимости экспериментальной подгонки и позволить использовать только параметры, выводимые из уравнений состояния (в частности, SAFT). Основной целью стало создание аналитической корреляции, не требующей знания вязкости как входного параметра, но использующей только предсказуемые или известные характеристики вещества.
      
      \textbf{Структура модели.}  
      Модель MYS представляет собой сложную аналитическую формулу, включающую молекулярные параметры, полученные из теории SAFT:
      
      \begin{equation}
        \eta = 0.1 \left( \exp \left( \frac{c_1 \sqrt{m} + \ln \left( 1 + \frac{M_w^4}{c_2 v^4 m^3} \right) \ln \left( \frac{T}{T_{\text{tp}} + 120} \right)}{
        \exp \left( \frac{1.04 v}{N_{\text{av}} m \sigma^3} \exp \left( \frac{c_3 P}{\epsilon_k \sqrt{-\frac{dP}{dv} \cdot \sqrt{m v}}} \right) - 1 \right) - 1} 
        \right) - 1 \right) + \eta_0
        \label{eq:mys}
      \end{equation}
      
      где:  
      \begin{itemize}
        \item $\eta$ — динамическая вязкость (Па·с)
        \item $\eta_0$ — вязкость идеального газа (обычно берется как постоянная или ноль)
        \item $m$ — молярная масса
        \item $M_w$ — молекулярная масса
        \item $v$ — мольный объем
        \item $T$ — температура
        \item $T_{\text{tp}}$ — температура тройной точки
        \item $P$ — давление
        \item $\epsilon_k$ — параметр межмолекулярного взаимодействия
        \item $dP/dv$ — производная давления по объему (рассчитанная из уравнения состояния)
        \item $\sigma$ — эффективный диаметр молекулы
        \item $N_{\text{av}}$ — число Авогадро
        \item $c_1=0.27$, $c_2=2.5*10^{11}$, $c_3=2.1$ — эмпирические коэффициенты (подобраны автором)
      \end{itemize}
      
      \textbf{Особенности модели.}  
      - Формула MYS не требует априорного знания вязкости, но использует величины, производные из термодинамических функций, так как она связана с параметрами уравнения состояния CP-PC-SAFT.
      - Автор демонстрирует среднюю ошибку модели на уровне менее 10\% для широкой выборки жидкостей.
      - Подход особенно эффективен для органических и углеводородных жидкостей.
      
      \textbf{Применение в данной работе.}  
      Модель MYS была выбрана в качестве основного ориентира для сравнения, поскольку:
      \begin{itemize}
        \item Она является одной из наиболее точных универсальных формул, не требующих подбора индивидуальных коэффициентов для каждого вещества.
        \item Она опирается на термодинамически интерпретируемые параметры, аналогично нашему подходу.
        \item Ее точность считается приемлемой для инженерных и прикладных задач.
        \item Она сохраняет свою точность на достаточно большом диапазоне давлений и температур
      \end{itemize}
      
      В настоящем исследовании модели машинного обучения и символьной регрессии сравниваются по точности предсказания вязкости с моделью MYS. Наши модели демонстрируют в ряде случаев ошибку до 2\% — значительно ниже, чем у MYS — что позволяет рассматривать их как более точную альтернативу в условиях, когда имеются данные для обучения модели.
      
    \subsubsection{Уравнение состояния CP-PC-SAFT}
    
    Основное уравнение состояния PC-SAFT выражается через вклад свободной энергии Гельмгольца \( a \):
    
    \begin{equation}
    A(v, T) = A^{\text{id}} + A^{\text{hs}} + A^{\text{chain}} + A^{\text{disp}} + A^{\text{assoc}}
    \end{equation}
    
    Где:
    \begin{itemize}
        \item \( A^{\text{id}} \) — вклад идеального газа;
        \item \( A^{\text{hs}} \) — вклад жёстких сфер;
          \[
          A^{HS} = RT \frac{m}{\zeta_0} \left( \frac{3 \zeta_1 \zeta_2}{1 - \zeta_3} + \frac{\zeta_2^3}{\zeta_3 (1 - \zeta_3)^2} \right) + \left( \frac{\zeta_2^3}{\zeta_3^2} - \zeta_0 \right) \ln [1 - \zeta_3] \sqrt{\frac{d^3 (\zeta_3 - 1)}{\zeta_3 \sigma^3 - d^3}}
          \]
        \item \( A^{\text{chain}} \) — вклад цепных взаимодействий;
          \[
          A^{chain} = RT \sum_i x_i x_j (1 - m_{ij}) \ln [g_{ij} (d_{ij})]^{hs}
          \]
        \item \( A^{\text{disp}} \) — вклад дисперсионных сил;
          \begin{align}
          A^{\text{disp}} = -RN_{\text{Av}} \Bigg( & \frac{2 \pi (\epsilon / k) m^2 \sigma^3}{\nu} I_1 \notag \\
          & + \frac{\pi (\epsilon / k)^2 m^3 \sigma^3}{\nu T \left(
          1 + \frac{m (8 \zeta_3 - 2 \zeta_3^2)}{(1 - \zeta_3)^4} +
          \frac{(1 - m)(20 \zeta_3 - 27 \zeta_3^2 + 12 \zeta_3^3 - 2 \zeta_3^4)}{((1 - \zeta_3)(2 - \zeta_3))^2}
          \right)} I_2 \Bigg)
          \end{align}
    
        \item \( A^{\text{assoc}} \) — вклад ассоциативных взаимодействий.
        \item $d$ — эффективный диаметр молекулы, связанный с $\sigma$ через $\theta$.
          \[
            d = \theta * \sigma
          \]
        \item $\sigma$ — характерный размер молекулы (диаметр сферического сегмента).
        \item $\theta$ — коэффициент масштабирования длины связи.
          \[
          \theta = \frac{1 + 0.2977 (k / \epsilon) T}{1 + 0.33163 (k / \epsilon) T + 0.0010477 (k / \epsilon)^2 T^2}
          \]
        \item $\zeta_k$ — параметр упаковки молекул.
          \[
          \zeta_k = \frac{\pi N_{av}}{6 \nu} \sum_i x_i m_i \sigma_i^{dk}
          \]
        \item $\epsilon / k_B$ — приведённая энергия взаимодействия между частицами.
        \item $N_{Av}$ — число Авогадро.
        \item $m$ — количество сегментов в молекуле.
        \item $x_i$ — мольная доля компонента $i$.
        \item $P$ — давление.
        \item $T$ — температура.
        \item $V_m$ — мольный объём.
        \item $d_{ij}$ — средний эффективный диаметр молекул $i$ и $j$.
        \item $g_{ij}$ — функция радиальной корреляции молекул $i$ и $j$.
        \item $\nu$ — молярный объём системы.
        \item $I_1$, $I_2$ — интегралы в дисперсионном вкладе.
    
    \end{itemize}
    
    \subsubsection{Параметры веществ  в CP-PC-SAFT}

В рамках CP-PC-SAFT каждое вещество описывается тремя основными параметрами:
\begin{itemize}
    \item \( m \) — число сегментов в молекуле;
    \item \( \sigma \) — эффективный диаметр сегмента молекулы;
    \item \( \varepsilon \) — энергия взаимодействия между сегментами.
\end{itemize}

Эти параметры определяются через критическую точку и температуру кипения вещества, а не через эмпирическую подгонку, что делает модель более универсальной.

В модели CP-PC-SAFT параметры вещества корректируются так, чтобы соответствовать следующим условиям \cite{polishuk2014standardized}:

\begin{equation}
\left( \frac{\partial P}{\partial v} \right)_{T_c} = 0, \quad
\left( \frac{\partial^2 P}{\partial v^2} \right)_{T_c} = 0
\end{equation}

\begin{equation}
P_c = P_{\text{c, exp}}
\end{equation}

\begin{equation}
\rho_{\text{liq, triple}} = \rho_{\text{liq, triple, exp}}
\end{equation}

Здесь:
\begin{itemize}
    \item \( P_c \) — давление в критической точке;
    \item \( T_c \) — критическая температура;
    \item \( \rho_{\text{liq, triple}} \) — жидкостная плотность в тройной точке.
\end{itemize}

Условиями для поиска параметров являются критическая температура, критическое давление и плотность жидкости при температуре плавления, соответствующие экспериментальным значениям. Необходимость знания значений всего трех экспериментальных величин для вычисления всех параметров вещества делает модель CP-PC-SAFT крайне привлекательной в вопросах предсказания свойств новых веществ и смесей на большом диапазоне условий. 

\subsubsection{Трудности машинного обучения}

Применение методов машинного обучения (ML) к задачам физико-химического моделирования, включая предсказание вязкости жидкостей, в последние годы стало популярным направлением (for example). Такие методы, как правило, позволяют автоматически выявлять сложные зависимости в данных и достигать высокой точности без необходимости ручного выбора аналитических формул. Тем не менее, несмотря на прогресс в области ML, существует ряд фундаментальных причин, по которым машинное обучение нельзя считать универсальным решением (панацеей) для предсказания реологических свойств веществ.

\textbf{1. Ограниченная интерпретируемость моделей.}
Большинство моделей машинного обучения, включая градиентный бустинг, нейронные сети и случайные леса, представляют собой черные ящики, в которых интерпретация внутренней логики и значимости отдельных признаков затруднена. Это особенно критично в контексте задач молекулярной термодинамики, где важна не только точность, но и возможность объяснить результат через физически обоснованные параметры. В противоположность этому, аналитические формулы, такие как MYS, или, потенциально, символьная регрессия (используемая в данной работе) позволяют сохранять интерпретируемость и анализировать вклад каждой переменной.

\textbf{2. Риск переобучения и проблемы обобщения.}
Как показано в ходе данной работы, даже простые модели, такие как линейная регрессия или случайный лес, демонстрируют высокий риск переобучения при специальном разбиении выборки. В частности, точность существенно снижается при валидации на новых веществах, не представленных в обучающей выборке. Особенно ярко это проявляется в случае веществ, отличающихся по агрегатному состоянию — например, бутана. Это указывает на ограниченную способность моделей ML к интерполяции за пределами обучающих данных.

\textbf{3. Зависимость от качества и полноты данных.}
Модели машинного обучения требуют большого количества однородных и хорошо размеченных данных. В случае термодинамических величин, таких как вязкость, часто доступны только ограниченные экспериментальные измерения, сделанные в различающихся условиях и с неоднородной точностью. Это может как ограничивать возможности ML, делая необходимым использование большего количества уникальных производных признаков, полученных, например, из уравнений состояния, так и приводить к переобучению из-за попыток модели подстроиться к условиям конкретного эксперимента.

\textbf{4. Отсутствие устойчивости к физическим границам.}
Некоторые модели ML могут выдавать предсказания, нарушающие физические ограничения — например, отрицательную вязкость или аномальные значения при экстремальных температурах и давлениях. Это особенно заметно в моделях, не обученных на таких данных. В отличие от этого, аналитические формулы и интерпретируемые модели могут быть сконструированы с учетом физических асимптотик и ограничений.

\textbf{5. Практические требования к применению.}
Для внедрения моделей в инженерную практику важны не только высокая точность, но и компактность, воспроизводимость и простота вычислений. Модели машинного обучения, особенно глубокие нейросети, могут быть ресурсоемкими и плохо воспроизводимыми. Символьные формулы или регрессии, напротив, могут быть легко реализованы даже в простых инженерных расчетах.

Таким образом, несмотря на мощность и гибкость современных методов машинного обучения, они должны использоваться с осторожностью, особенно в задачах с физико-химическим контекстом. Наиболее перспективным направлением видится интеграция ML с физически обоснованными подходами — в частности

\subsubsection{Символьная регрессия}

Символьная регрессия (symbolic regression, SR) — это подход к построению моделей, при котором алгоритм не просто находит численные параметры в заданной структуре формулы как в линейной регрессии, а ищет саму структуру уравнения, наиболее точно описывающего зависимость между входными и выходными переменными. В отличие от традиционных методов машинного обучения, символьная регрессия формирует аналитическое выражение в виде комбинаций математических операций и входных признаков.

\textbf{Мотивация применения.}
Основным преимуществом SR является высокая интерпретируемость получаемых моделей. Вместо подбора большого количества слабоинтерпретируемых коэффициентов, как в случае с нейросетями, модель находит короткие выражения, состоящие из качественно различающихся блоков (операторов). Каждое уравнение, сформированное таким способом, может быть проанализировано, сопоставлено с физическими законами, а также непосредственно использовано в инженерных расчетах. Это особенно важно при работе с термодинамическими и молекулярными свойствами, где значение имеет не только точность, но и физический смысл модели.

В рамках настоящего исследования SR использовалась для поиска наиболее компактных и точных выражений, связывающих вязкость с параметрами уравнения состояния CP-PC-SAFT и производными термодинамическими величинами, включая мольный объем, избыточную энтропию и производную давления по объему.

\textbf{Используемый инструмент: PySR.}
Для реализации символьной регрессии применялась современная библиотека PySR (Python Symbolic Regression) \cite{cranmer2023pysr}. Эта библиотека сочетает возможности языка Python для настройки задач и быстродействие языка Julia, где реализована основная эволюционная оптимизация. PySR использует алгоритмы на основе генетического программирования, направленные на поиск выражений с оптимальным компромиссом между точностью и сложностью.

Основные особенности PySR:
\begin{itemize}
  \item Многофункциональная система операторов: поддерживаются стандартные математические функции (логарифм, экспонента, степень и др.).
  \item Поддержка многоцелевой оптимизации: минимизация ошибки при контроле за длиной формулы.
  \item Устойчивость к переобучению: регуляризация сложности встроена в процесс поиска.
  \item Высокая производительность благодаря параллельному выполнению на CPU и GPU.
\end{itemize}

\textbf{Результаты применения.}
В ходе экспериментов символьная регрессия позволила получить компактные формулы с ошибкой, сравнимой с ошибкой модели MYS. Таким образом, символьная регрессия показала себя как перспективный инструмент, позволяющий извлекать простые закономерности. Ее применение может дополнить использование других моделей и усилить акцент на физическую обоснованность результатов.

  \subsection{Цель и задачи исследования}

    \subsubsection{Цель исследования}
    
    Разработка метода прогнозирования вязкости жидкостей на основе параметров уравнения состояния CP-PC-SAFT с точностью, превосходящей существующую модель MYS.
    
    \subsubsection{Задачи исследования}
    
    Для реализации постасленной цели были намечены следующие задачи:
    
    \begin{enumerate}
      \item \textbf{Сбор и подготовка экспериментальных данных}{
        \begin{itemize}
          \item Получение доступа к данным термодинамических параметров веществ.
          \item Формирование базы данных свойств жидкостей и газов на основе ThermoML.
          \item Фильтрация выборки для получения только интересующих веществ с замеренными температурой, давлением и вязкостью.
          \item Предварительная обработка данных для приведения интересующих параметров к единому формату.
        \end{itemize}
      }
    
        \item \textbf{Расчет производных параметров}
        \begin{itemize}
            \item Вычисление мольного объема по известным температуре и давлению.
            \item Определение избыточной энтропии $S^{\text{ex}}$ по известным мольному объему и температуру через производную избыточной энергии Гельмгольца по температуре.
            \item Вычисление вязкости идеального газа.
        \end{itemize}
    
        \item \textbf{Генерация и отбор новых признаков, проверка их значимости}
        \begin{itemize}
            \item Создание производных признаков, отражающих физические закономерности вязкости.
            \item Оценка их статистической значимости для модели, и отбор наиболее полезных.
            \item Валидация роли различных признаков в прогнозировании вязкости, в том числе избыточной энтропии.
        \end{itemize}
    
        \item \textbf{Обучение, сравнение и оптимизация моделей машинного обучения}
        \begin{itemize}
            \item Построение и проверка эффективности различных моделей ML, таких как линейная регрессия, градиентный бустинг, случайный лес и нейросетевые методы.
            \item Оптимизация гиперпараметров для повышения точности наиболее перспективных моделей.
            \item Использование символьной регрессии (PySR) для поиска интерпретируемых аналитических зависимостей.
        \end{itemize}
    
        \item \textbf{Сравнение разработанной модели с существующей моделью MYS}
        \begin{itemize}
            \item Оценка точности предложенного метода на тестовых данных.
            \item Анализ преимуществ и недостатков разработанных моделей по сравнению с MYS.
            \item Формирование рекомендаций по дальнейшему развитию метода прогнозирования вязкости жидкостей.
        \end{itemize}
    \end{enumerate}
  
  \subsection{Выводы по главе}
В первой главе был проведён обзор современных представлений о вязкости как важном термодинамическом параметре и существующих подходах к её моделированию. Показано, что вязкость оказывает существенное влияние на поведение жидких и многофазных систем в различных инженерных приложениях — от нефтепереработки до микрофлюидики.

Рассмотрены эмпирические формулы, применяемые для расчёта вязкости, а также универсальные аналитические модели, такие как MYS (Modified Yarranton–Satyro). Несмотря на широкую применимость первых, они имеют ограничения в точности при отсутствии экспериментальных данных для подгонки или при переходе к новым веществам. В свою очередь, MYS не всегда дает достаточные для прикладных задач оценки ввиду своей универсальности.  

Особое внимание было уделено уравнению состояния CP-PC-SAFT, которое позволяет получить необходимые параметры вещества без тщательной подгонки, опираясь лишь на критические и тройные точки. Это делает его особенно подходящим для предсказательных моделей, работающих на широком спектре условий и веществ, таких как MYS или разрабатываемая в рамках данной работы.

Кроме того, рассмотрены возможности машинного обучения и его ограничения в задачах физико-химического моделирования, включая проблемы интерпретируемости, переобучения и нарушения физических ограничений. На этом фоне была обоснована целесообразность использования символьной регрессии и системы генерации новых признаков как методов, сочетающих гибкость ML с интерпретируемостью и физической обоснованностью аналитических формул.

Таким образом, глава обосновывает научную и практическую значимость задачи предсказания вязкости на основе параметров уравнения состояния и подтверждает пользу разработки нового подхода, сочетающего машинное обучение и следование физическим принципам. Далее в работе рассматривается реализация этого подхода на практике.

\newpage

% Глава 2
\section{Сбор и обработка данных}
  \subsection{Работа с базой данных}
    \subsubsection{Источник данных: база ThermoML}
    
      Для получения экспериментальных данных о вязкости в данной работе использовалась база данных \textbf{ThermoML}, разработанная Национальным институтом стандартов и технологий США (NIST). ThermoML представляет собой машиночитаемый формат на основе XML, предназначенный для хранения и распространения термодинамических и физико-химических данных, включая фазовые равновесия, плотности, давления, температуры и вязкости чистых веществ и смесей.
      
      Одной из ключевых особенностей ThermoML является строго стандартизированная структура описания данных, что позволяет автоматизировать их извлечение и минимизировать необходимость ручной предобработки. Это особенно важно при формировании выборок для задач машинного обучения, где требуется согласованность единиц измерения и форматов.
      
    \subsubsection{ThermoPyL для чтения данных}
      Для взаимодействия с базой ThermoML была использована специализированная Python-библиотека \textbf{ThermoPyL}. Данная библиотека предоставляет инструменты для:
      \begin{itemize}
        \item автоматического чтения и парсинга XML-файлов ThermoML;
        \item фильтрации данных по фазе и составу;
        \item преобразования данных в формат \texttt{pandas.DataFrame} для последующей обработки в Python;
        \item агрегации данных по условиям экспериментов;
        \item экспорта результатов в формат CSV.
      \end{itemize}
      
      Использование библиотеки \texttt{ThermoPyL} позволило автоматизировать процесс подготовки данных, не углубляясь в анализ и обработку XML файлов из ThermoML.
    
  \subsection{Выбор веществ и критерии включения}
  
  Для анализа были выбраны чистые вещества, относящиеся к классу насыщенных углеводородов. Основным критерием отбора было наличие значительного объёма экспериментальных данных по вязкости в базе ThermoML. В финальную выборку вошли восемь веществ:
    \begin{itemize}
      \item Бутан: получено 311 экспериментальных точек из 2 работ \cite{acs.jced.5b00654,s10765-006-0053-2}
      \item Пентан: получено 50 экспериментальных точек из 3 работ \cite{acs.jced.7b00650,j.jct.2005.10.011,je0501944}
      \item Гексан: получено 127 экспериментальных точек из 25 работ \cite{acs.jced.5b00152,acs.jced.8b00589,j.fluid.2013.07.060,j.fluid.2018.08.001,j.jct.2004.09.021,j.jct.2005.03.024,j.jct.2005.07.008,j.jct.2005.10.011,j.jct.2007.05.016,j.jct.2008.02.005,j.tca.2005.10.008,je020114l,je020131a,je030107c,je034002l,je049576k,je049777o,je060389r,je060491o,je800048f,je8006138,je9006597,s10765-005-5572-8,s10765-006-0053-2,s10765-009-0622-2}
      \item Гептан: получено 443 экспериментальных точек из 28 работ \cite{acs.jced.5b00152,acs.jced.7b00121,j.fluid.2006.01.030,j.fluid.2010.10.009,j.fluid.2016.11.029,j.jct.2004.09.021,j.jct.2005.03.024,j.jct.2005.10.011,j.jct.2006.01.012,j.jct.2013.09.017,j.jct.2015.12.021,j.tca.2005.10.008,j.tca.2015.08.005,j.tca.2018.10.018,je020131a,je049662k,je049776w,je049777o,je400835u,je5007532,je600554h,je700202h,je8003707,je9006597,je900969u,s10765-009-0667-2,s10765-012-1373-z,s10765-014-1759-1}
      \item Октан: получено 362 экспериментальных точек из 32 работ \cite{acs.jced.6b00391,acs.jced.7b00121,acs.jced.7b00650,j.fluid.2010.10.009,j.jct.2003.12.005,j.jct.2004.09.021,j.jct.2005.03.024,j.jct.2005.10.011,j.jct.2006.01.012,j.jct.2007.05.016,j.jct.2008.02.005,j.jct.2013.09.017,j.jct.2014.09.015,j.tca.2005.10.008,j.tca.2015.08.005,je020131a,je034017j,je034208m,je049572f,je049776w,je0503296,je060389r,je300899n,je4004806,je400835u,je600554h,je800348s,je800417q,je9006597,s10765-006-0053-2,s10765-008-0542-6,s10765-014-1759-1}
      \item Нонан: получено 115 экспериментальных точек из 13 работ \cite{acs.jced.7b00121,acs.jced.7b00650,j.fluid.2010.10.009,j.jct.2004.09.021,j.jct.2005.03.024,j.jct.2005.10.011,j.jct.2013.09.017,je200757a,je300899n,je400250u,je400835u,je700202h,je700428f}
      \item Декан: получено 224 экспериментальных точек из 25 работ \cite{acs.jced.5b00270,acs.jced.6b00542,acs.jced.7b00121,acs.jced.7b00650,j.fluid.2010.10.009,j.jct.2003.12.005,j.jct.2004.09.021,j.jct.2005.03.024,j.jct.2005.10.011,j.jct.2007.05.016,j.jct.2008.02.005,j.jct.2013.05.014,j.jct.2013.09.017,je034208m,je0501585,je060389r,je300899n,je400250u,je400835u,je4008926,je800348s,je800417q,je800635g,s10765-005-5572-8,s10765-008-0542-6}
      \item Додекан: получено 412 экспериментальных точек из 25 работ \cite{acs.jced.6b00391,acs.jced.6b00542,acs.jced.6b00688,acs.jced.7b00201,acs.jced.7b00466,acs.jced.7b00650,acs.jced.7b00866,acs.jced.8b00008,acs.jced.8b00438,acs.jced.8b01135,acs.jced.8b01233,acs.jced.9b00187,j.fluid.2015.07.022,j.jct.2003.12.005,j.jct.2004.09.021,j.jct.2005.03.024,j.jct.2013.12.022,j.jct.2014.02.012,j.jct.2015.12.021,je034208m,je060491o,je200757a,je400493x,je400835u,je5000132}
    \end{itemize}
    
    Для включения записи в выборку требовалось наличие всех трёх параметров:
    \begin{itemize}
      \item температуры (\( T \), К),
      \item давления (\( P \), Па),
      \item вязкости (\( \eta \), Па·с).
    \end{itemize}
    
    Записи, в которых отсутствовала хотя бы одна из этих величин, исключались. Дополнительная фильтрация по фазе не производилась, поскольку, например, для бутана полезны данные как в жидкой, так и в газообразной фазе. Каждая выборка сохранялась в виде отдельного CSV-файла для дальнейшей обработки. В результате было собрано более 2000 уникальных записей.

  \subsection{Распределение данных}
    \subsubsection{Фазовое состояние}
      Большая часть данных принадлежала к жидкой фазе. Только 295 точек для бутана находятся в состоянии газа. Так как модель MYS расчитана на данные как в жидкой, так и в газовой фазе, было решено оставить точки, принадлежащие газовой фазе, несмотря на то, что они представлены только бутаном.

    \subsubsection{Температура}
      Значения температуры итоговых данных лежат в диапазоне от 273.15 до 693.7 Кельвинов с явным пиком в районе 300 Кельвинов. Средняя температура данных - 406 Кельвинов, а медианная 403.
      \begin{figure}[ht]
        \centering
        \includegraphics[width=0.8\textwidth]{data_distribution_temperature.png}
        \caption{Распределение температуры в экспериментальных данных}
        \label{fig:data_distribution_temperature}
      \end{figure}

    \subsubsection{Давление}
      Значения давления итоговых данных лежат в диапазоне от 57.183 кПа до 245.16 МПа. Данные удобнее отображать в логарифмической шкале, так как в ней они относительно равномерно распределены за исключением резкого пика в районе 1 атмосферы. Среднее значение давления для экспериментальных данных - 23 МПа, медианное - 6 МПа.
      \begin{figure}[ht]
        \centering
        \includegraphics[width=0.8\textwidth]{data_distribution_pressure.png}
        \caption{Распределение давления в экспериментальных данных (логарифмическая шкала по горизонтали)}
        \label{fig:data_distribution_pressure}
      \end{figure}

    \subsubsection{Вязкость}
      Значения вязкости итоговых данных лежат в диапазоне от 7.3 мкПа*с до 5.8 мПа*с. Распределение вязкости, аналогично распределению давления, удобнее отображать в логарифмическом масштабе. Данные имеют небольшой пик в районе 10 мкПа*с за счет данных по экспериментам с бутаном, где вещество находится в газовой фазе. Среднее значения вязкости для экспериментальных данных - 500 мкПа*с, медианное - 385 мкПа*с. 
      \begin{figure}[ht]
        \centering
        \includegraphics[width=0.8\textwidth]{data_distribution_viscosity.png}
        \caption{Распределение вязкости в экспериментальных данных (логарифмическая шкала по горизонтали)}
        \label{fig:data_distribution_viscosity}
      \end{figure}

  \subsection{Добавление параметров уравнения CP-PC-SAFT}
    Как уже было упомянуто выше, у каждого вещества есть единственный набор параметров для уравнения CP-PC-SAFT: $\sigma$, $\delta$, $\epsilon/k_b$ и $m$. Для всех исследуемых веществ данные признаки были добавлены в датасет. Эти параметры представляют собой ключевые свойства каждого соединения, поскольку их можно точно определить, что позволяет назвать их основными характеристиками.
    \begin{table}[ht]
      \TableNumberRight

      \begin{center}
        \textbf{Основные характеристики компонентов}
        \vspace*{\fill}
      \end{center}

      \vspace{0.8ex}
      
      % Table aligned left + full width
      \noindent
      \begin{tabular}{|l|S|S|S|S|S|}
          \hline
          \textbf{Вещество} & \textbf{Молярная масса, (г/моль)} & \textbf{m} & \textbf{$\epsilon/k_b$, K} & \textbf{\sigma, \si{\angstrom} } & \textbf{\delta} \\
          \hline
          butane & 58.120 & 2.48262 & 209.446 & 3.65040 & 1.15976 \\
          pentane & 72.150 & 3.06424 & 212.528 & 3.62421 & 1.16385 \\
          hexane & 86.180 & 3.51081 & 218.238 & 3.65575 & 1.16091 \\
          heptane & 100.210 & 4.07032 & 220.494 & 3.63515 & 1.16631 \\
          octane & 114.230 & 4.45475 & 225.287 & 3.67868 & 1.17934 \\
          nonane & 128.257 & 4.85100 & 229.271 & 3.70469 & 1.18722 \\
          decane & 142.284 & 5.27013 & 232.262 & 3.72188 & 1.20260 \\
          dodecane & 170.338 & 6.01209 & 238.240 & 3.76983 & 1.22522 \\
          \hline
      \end{tabular}
      \label{tab:components_data}
    \end{table}

  \subsection{Вычисление производных величин}
    В ходе работы возникала потребность в дополнении данных новыми величинами, которые можно вычислить используя уже известные величины. Это нужно как для увеличения количества признаков, которыми могли бы пользоваться модели машинного обучения, так и для возможности пользоваться некоторыми формулами.
    
    \subsubsection{Вязкость идеального газа}
      Вязкость идеального газа является одним из параметров уравнения MYS~\eqref{eq:mys}. Можно было бы опустить этот коэффициент и использовать вместо него ноль, и формально формула осталась бы вполне точной для жидкой фазы. Однако для газовой области именно этот коэффициент даёт наибольший вклад. Пренебречь им — значит закрыть глаза на половину картины. Поэтому было решено вычислить вязкость идеального газа для всех экспериментальных точек, чтобы обеспечить максимальную точность модели в целом. Поскольку уравнение MYS служит основным критерием точности, задача вычисления $\eta_0$ приобрела особую значимость.

      Для расчёта вязкости идеального газа использовался классический подход на основе параметров Леннарда--Джонса — $\epsilon/k$ и $\sigma$ (for example), которые были определены по температуре плавления и мольному объёму вещества. Этот подход, несмотря на свою простоту, оказывается весьма работоспособным при корректной выборке входных данных. Температура плавления ($T_m$) и плотность в жидком состоянии ($\rho$) использовались для расчёта мольного объёма $V_m$ по формуле:
      \[
      V_m = \frac{M}{\rho},
      \]
      где $M$ — молярная масса.
      
      Далее параметры $\epsilon/k$ и $\sigma$ определялись следующим образом:
      \[
      \frac{\epsilon}{k} = 1.92 \cdot T_m, \quad 
      \sigma = \left( \frac{2.3 \cdot V_m}{\pi N_A \cdot 2/3} \right)^{1/3},
      \]
      где $N_A$ — число Авогадро. Стоит отметить, что здесь $V_m$ переводился в кубические метры, а итоговое значение $\sigma$ — в ангстремы.
     
      \begin{table}[ht]
        \TableNumberRight
      
        \begin{center}
          \textbf{Параметры Леннарда--Джонса для выбранных алканов (for example)}
          \vspace*{\fill}
        \end{center}
      
        \vspace{0.8ex}
      
        \noindent
        \begin{tabular}{|l
                |S[round-mode=places,round-precision=2]
                |S[round-mode=places,round-precision=4]
                |S[round-mode=places,round-precision=2]
                |S[round-mode=places,round-precision=4]|}

          \hline
          \textbf{Вещество} & {\textbf{$\epsilon/k_b$, K}} & {\textbf{$\sigma$, \si{\angstrom}}} & {\textbf{$\epsilon/k_b$ lit., K}} & {\textbf{$\sigma$ lit., \si{\angstrom}}}\\
          \hline
          butane     & 410.00 & 4.9970 & 257.28 & 5.7016 \\
          pentane    & 345.00 & 5.7690 & 274.56 & 5.9456 \\
          hexane     & 413.00 & 5.9090 & 341.76 & 6.2013 \\
          heptane    & 350.59 & 6.4406 & 350.59 & 6.4406 \\
          octane     & 345.00 & 5.7690 & 415.30 & 6.6668 \\
          nonane     & 413.00 & 5.9090 & 422.40 & 6.8806 \\
          decane     & 467.52 & 7.0836 &  &  \\
          dodecane   & 506.11 & 7.4540 &  &  \\
          \hline
        \end{tabular}
        \label{tab:lj_params}
      \end{table}

      Полученные параметры Леннарда--Джонса использовались далее для расчёта вязкости идеального газа. Выражение для вязкости, выведено в рамках кинетической теории газов и может быть найдено в продвинутых книгах по молекулярной динамике \cite{molecular1955theory}и включает поправки на столкновения. Финальное выражение имеет следующий вид:
      \[
      \eta_0 = \frac{5}{16} \cdot \frac{\sqrt{MRT}}{\pi^{1/2} \cdot \sigma^2 \cdot \Omega(T^*) \cdot N_A},
      \]
      где $T^*$ — приведённая температура, а $\Omega$ — коэффициент столкновений, зависящий от $T^*$:
      \[
      T^* = \frac{RT}{\epsilon/k_b}, \quad
      \Omega(T^*) = \frac{A}{(T^*)^B} + C e^{-D T^*} + E e^{-F T^*} + R T^{*B} \sin(S T^{*W} - P).
      \]
      Ниже приведены значения эмпирических коэффициентов, использованных в выражении для функции столкновений $\Omega(T^*)$. Эти значения были взяты из работы \cite{neufeld1972collision} и использованы без модификации.
      \begin{table}[ht]
        \TableNumberRight
        \begin{center}
          \textbf{Коэффициенты для расчёта функции столкновений $\Omega(T^*)$}
          \vspace*{\fill}
        \end{center}
      
        \vspace{0.8ex}
        \noindent
        \begin{tabular}{|c|S|}
          \hline
          \textbf{Коэффициент} & {\textbf{Значение}} \\
          \hline
          $A$ & 1.16145 \\
          $B$ & 0.14874 \\
          $C$ & 0.52487 \\
          $D$ & 0.77320 \\
          $E$ & 2.16178 \\
          $F$ & 2.43787 \\
          $R$ & -0.0006435 \\
          $S$ & 18.0323 \\
          $W$ & -0.76830 \\
          $P$ & 7.27371 \\
          \hline
        \end{tabular}
        \label{tab:neufeld_coeffs}
      \end{table}

      
      Расчёты выполнялись для каждой строки из экспериментальной выборки с помощью функции `apply()` библиотеки `pandas`. Итоговая вязкость $\eta_0$ вносилась в результирующий CSV-файл, на основе которого строилось дальнейшее сравнение с экспериментальными и модельными данными.      
      
    \subsubsection{Мольный объем}

    \subsubsection{Избыточная энтропия}


% Основная идея заключается в том, что логарифм безразмерной вязкости жидкости может быть представлен как функция избыточной энтропии вещества, нормализованной по постоянной Больцмана. Таким образом, в основу модели заложена физически обоснованная связь между вязкостью и степенью упорядоченности молекулярной структуры жидкости, выраженной через избыточную энтропию:
% 
% \[
% \ln \left( \frac{\eta}{\eta_0} \right) = A + B \cdot \frac{S^{\text{ex}}}{k_B} + C \cdot \left(\frac{S^{\text{ex}}}{k_B} \right)^2 + \ldots
% \]
% 
% где:
% - \( \\eta \) — динамическая вязкость жидкости,
% - \( \\eta_0 \) — характеристическая (референсная) вязкость,
% - \( S^{\\text{ex}} \) — избыточная энтропия,
% - \( A, B, C \) — подбираемые параметры.


  \subsection{Выводы по главе}
\newpage

% Глава 3
\section{Построение моделей и генерация признаков}
  \subsection{Автоматическая генерация признаков}
  \subsection{Отбор признаков}
  \subsection{Символьная регрессия}
  \subsection{Проверка значимости избыточной энтропии}
  \subsection{Выводы по главе}
\newpage

% Глава 4
\section{Обучение моделей и анализ результатов}
  \subsection{Обучение моделей ML}
  \subsection{Сравнение подходов}
  \subsection{Сравнение с моделью MYS}
  \subsection{Выводы по главе}
\newpage

% Заключение
\section*{Заключение}
\addcontentsline{toc}{section}{Заключение}
(Основные выводы и перспективы)
\newpage

% Глоссарий (необязательно)
% \section*{Глоссарий}
% \addcontentsline{toc}{section}{Глоссарий}
% (Определения терминов)

% Список сокращений (необязательно)
% \section*{Список сокращений}
% \addcontentsline{toc}{section}{Список сокращений}
% (Обозначения и сокращения)

% Список литературы
% \section*{Список использованных источников}
\addcontentsline{toc}{section}{Список литературы}
\printbibliography

% Приложения
% \appendix
% \section{Пример формул}
% (Пример приложения)

\end{document}


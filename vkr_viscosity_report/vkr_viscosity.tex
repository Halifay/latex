\documentclass[a4paper,12pt]{article}

\usepackage{fontspec}
\usepackage{polyglossia}
\setmainlanguage{russian}
\setotherlanguage{english}

% Set main font (Latin)
\setmainfont{Times New Roman} % or any other Latin font you like

% Set font for Cyrillic script explicitly
\newfontfamily\cyrillicfont{Times New Roman} % this font supports Cyrillic on most systems
\newfontfamily\cyrillicfonttt{Courier New} % Or another monospace font that supports Cyrillic

\usepackage{amsmath, amssymb, amsfonts, graphicx, csquotes, array, geometry, titlesec, biblatex, listings, hyperref, caption, xcolor, import, subcaption}

\addbibresource{references.bib} % Файл со списком литературы

\geometry{left=25mm,right=20mm,top=25mm,bottom=25mm}

% \hypersetup{colorlinks=false}
% \usepackage{tocloft}
% \renewcommand{\cftsecfont}{\normalfont}
% \renewcommand{\cftsecpagefont}{\normalfont}
\hypersetup{
	% linkcolor=white,
	linkbordercolor = white,          
	citebordercolor = green,
}

\setcounter{page}{2}

% Оглавление
% \renewcommand{\cftsecleader}{\cftdotfill{\cftdotsep}}

\title{\Huge{Предсказание вязкости жидкостей с помощью методов машинного обучения на основе параметров уравнения состояния}}
\author{Панов Михаил Федорович \\ Руководитель: Писарев Василий Вячеславович}
\date{Москва --- 2025}

\begin{document}

% Титульный лист
\maketitle
\thispagestyle{empty}
\newpage

% Задание на ВКР
\section*{Задание на ВКР}
% \addcontentsline{toc}{section}{Задание на ВКР}
(Текст задания на ВКР)

% Аннотация
\section*{Аннотация}
% \addcontentsline{toc}{section}{Аннотация}
(Аннотация на русском языке)

% Abstract
\section*{Abstract}
% \addcontentsline{toc}{section}{Abstract}
(Аннотация на английском языке)
\newpage

% Содержание
\tableofcontents
\newpage

% Введение
\section*{Введение}
\addcontentsline{toc}{section}{Введение}
(Введение: актуальность, цель, задачи, объект, структура)

% Глава 1
\section{Обзор литературы и постановка задачи}
  \subsection{Свойства вязкости и её роль в инженерных задачах}

    \subsubsection{Вязкость в промышленных приложениях}

Вязкость — ключевой реологический параметр, определяющий сопротивление жидкости деформации при сдвиге. Её значение напрямую влияет на гидродинамические характеристики систем, включая распределение давления, скорости потока и теплопередачу. В ряде инженерных процессов, таких как транспортировка нефти и газа, проектирование теплообменников, химическая переработка и производство смазочных материалов, точное знание вязкости определяет эффективность и безопасность эксплуатации оборудования. 

В многофазных или сложных многокомпонентных системах, например, в синтетических нефтяных смесях, вязкость влияет на фазовое поведение, устойчивость эмульсий и общую производительность процессов. В микрофлюидике и фармацевтике контроль вязкости необходим для предсказуемого управления потоками и дозированием веществ.

    \subsubsection{Значение точного прогнозирования вязкости}

Точное моделирование вязкости представляет собой одну из ключевых задач в термодинамическом моделировании жидкостей. На практике невозможно измерить вязкость для всех возможных условий и составов, особенно в системах, содержащих десятки или сотни компонентов, как, например, в нефтяной промышленности. В таких случаях используются предсказательные модели, основанные на молекулярных или термодинамических представлениях, которые позволяют получить значения вязкости при отсутствии экспериментальных данных.

Среди современных подходов особое внимание уделяется моделям, связывающим вязкость с остаточной энтропией и другими термодинамическими величинами, так как они обеспечивают высокую переносимость и физическую обоснованность. Связь между вязкостью и избыточной энтропией подмечали во многих работах, например \cite{taib2020residual}. Особенно актуальны такие модели для расчётов в широком диапазоне температур и давлений, а также при переходе к сверхкритическим и конденсированным фазам.

Таким образом, разработка универсальных моделей вязкости, не требующих индивидуальной подгонки параметров, является приоритетной задачей в области физико-химического моделирования сложных жидкостей.

  \subsection{Обзор подходов к предсказанию вязкости}

    \subsubsection{Ограничения эмпирических формул}
Большинство классических подходов к описанию вязкости основаны на эмпирических уравнениях, содержащих подобранные коэффициенты для конкретных веществ или их классов. Такие модели, как правило, работают в узком диапазоне температур и давлений и требуют предварительных измерений вязкости. Это делает их трудноприменимыми для новых или редких веществ, особенно в условиях, когда экспериментальные данные недоступны или ограничены. (for example)

    \subsubsection{Модель MYS}

Одной из наиболее универсальных и известных современных моделей для описания динамической вязкости жидкостей в широком диапазоне условий является модель, предложенная Ильёй Полищуком в 2015 году — так называемая \textit{Modified Yarranton–Satyro (MYS)} модель~\cite{polishuk2015viscosity}. Её основное назначение — обеспечить точное предсказание вязкости для широкого спектра органических жидкостей, включая углеводороды, в диапазонах температур от тройной точки до высоких значений, а также при изменяющемся давлении.

\textbf{Предпосылки модели.}  
Изначально модель Yarranton–Satyro (YS) создавалась для описания вязкости как функции плотности, температуры и параметров молекулярного строения. Polishuk модифицировал её, чтобы избежать необходимости экспериментальной подгонки и позволить использовать только параметры, выводимые из уравнений состояния (в частности, SAFT). Основной целью стало создание аналитической корреляции, не требующей знания вязкости как входного параметра, но использующей только предсказуемые или известные характеристики вещества.

\textbf{Структура модели.}  
Модель MYS представляет собой сложную аналитическую формулу, включающую молекулярные параметры, полученные из теории SAFT:

\[
\eta = 0.1 \left( \exp \left( \frac{c_1 \sqrt{m} + \ln \left( 1 + \frac{M_w^4}{c_2 v^4 m^3} \right) \ln \left( \frac{T}{T_{\text{tp}} + 120} \right)}{
\exp \left( \frac{1.04 v}{N_{\text{av}} m \sigma^3} \exp \left( \frac{c_3 P}{\epsilon_k \sqrt{-\frac{dP}{dv} \cdot \sqrt{m v}}} \right) - 1 \right) - 1} 
\right) - 1 \right) + \eta_0
\]

где:  
\begin{itemize}
  \item $\eta$ — динамическая вязкость (Па·с)
  \item $\eta_0$ — вязкость идеального газа (обычно берётся как постоянная или ноль)
  \item $m$ — молярная масса
  \item $M_w$ — молекулярная масса
  \item $v$ — мольный объём
  \item $T$ — температура
  \item $T_{\text{tp}}$ — температура тройной точки
  \item $P$ — давление
  \item $\epsilon_k$ — параметр межмолекулярного взаимодействия
  \item $dP/dv$ — производная давления по объёму (рассчитанная из уравнения состояния)
  \item $\sigma$ — эффективный диаметр молекулы
  \item $N_{\text{av}}$ — число Авогадро
  \item $c_1=$, $c_2$, $c_3$ — эмпирические коэффициенты (подобраны автором)
\end{itemize}

\textbf{Особенности модели.}  
- Формула MYS не требует априорного знания вязкости, но использует производные термодинамических функций, что делает её совместимой с уравнениями состояния, такими как PC-SAFT и CP-PC-SAFT.
- Автор демонстрирует среднюю ошибку модели на уровне менее 10\% для широкой выборки жидкостей.
- Подход особенно эффективен для органических и углеводородных жидкостей, а также используется как референтная модель в дальнейших работах (в том числе в настоящем исследовании).

\textbf{Применение в данной работе.}  
Модель MYS была выбрана в качестве основного ориентира для сравнения, поскольку:
\begin{itemize}
  \item Она является одной из наиболее точных универсальных формул, не требующих подбора индивидуальных коэффициентов для каждого вещества.
  \item Она опирается на термодинамически интерпретируемые параметры, аналогично нашему подходу.
  \item Её точность считается приемлемой для инженерных и прикладных задач.
\end{itemize}

В настоящем исследовании модели машинного обучения и символьной регрессии сравниваются по точности предсказания вязкости с моделью MYS. Наши модели демонстрируют в ряде случаев ошибку до 2\% — значительно ниже, чем у MYS — что позволяет рассматривать их как более точную альтернативу в условиях, когда известны параметры уравнения состояния.

    \subsubsection{Почему машинное обучение — это не панацея}
    (Краткий обзор ML-подходов — скорее как мотивация: сложность, переобучение, слабая интерпретируемость)

    \subsubsection{Интерпретируемый подход: генерация признаков и символьная регрессия}
    (контроль, физический смысл, устойчивость)

  \subsection{Цель и задачи исследования}
  \subsection{Выводы по главе}
\newpage

% Глава 2
\section{Сбор и обработка данных}
  \subsection{Источник данных: база ThermoML}
  \subsection{Выбор веществ и критерии включения}
  \subsection{Добавление параметров уравнения CP-PC-SAFT}
  \subsection{Вычисление производных величин}
    \subsubsection{Мольный объём}
    \subsubsection{Избыточная энтропия}

Основная идея заключается в том, что логарифм безразмерной вязкости жидкости может быть представлен как функция избыточной энтропии вещества, нормализованной по постоянной Больцмана. Таким образом, в основу модели заложена физически обоснованная связь между вязкостью и степенью упорядоченности молекулярной структуры жидкости, выраженной через избыточную энтропию:

\[
\ln \left( \frac{\eta}{\eta_0} \right) = A + B \cdot \frac{S^{\text{ex}}}{k_B} + C \cdot \left(\frac{S^{\text{ex}}}{k_B} \right)^2 + \ldots
\]

где:
- \( \\eta \) — динамическая вязкость жидкости,
- \( \\eta_0 \) — характеристическая (референсная) вязкость,
- \( S^{\\text{ex}} \) — избыточная энтропия,
- \( A, B, C \) — подбираемые параметры.


  \subsection{Выводы по главе}
\newpage

% Глава 3
\section{Построение моделей и генерация признаков}
  \subsection{Автоматическая генерация признаков}
  \subsection{Отбор признаков}
  \subsection{Символьная регрессия}
  \subsection{Проверка значимости избыточной энтропии}
  \subsection{Выводы по главе}
\newpage

% Глава 4
\section{Обучение моделей и анализ результатов}
  \subsection{Обучение моделей ML}
  \subsection{Сравнение подходов}
  \subsection{Сравнение с моделью MYS}
  \subsection{Выводы по главе}
\newpage

% Заключение
\section*{Заключение}
\addcontentsline{toc}{section}{Заключение}
(Основные выводы и перспективы)
\newpage

% Глоссарий (необязательно)
% \section*{Глоссарий}
% \addcontentsline{toc}{section}{Глоссарий}
% (Определения терминов)

% Список сокращений (необязательно)
% \section*{Список сокращений}
% \addcontentsline{toc}{section}{Список сокращений}
% (Обозначения и сокращения)

% Список литературы
% \section*{Список использованных источников}
\addcontentsline{toc}{section}{Список литературы}
\printbibliography

% Приложения
% \appendix
% \section{Пример формул}
% (Пример приложения)

\end{document}

